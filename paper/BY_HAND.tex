\documentclass[a4paper,debug]{tufte-handout}

\title{The cocycle: a toy version}
\date{\today}
\author{DMR}

\usepackage{amsmath}
\usepackage{diagbox}

\hypersetup{
urlcolor = red,
colorlinks = true,
linkcolor = blue,
citecolor = blue,
linktocpage = true,
pdftitle = {The cocycle: a toy version},
pdfauthor = {David M. Roberts},
bookmarksopen = false,
bookmarksopenlevel = 1,
unicode = true,
hypertexnames =false
}%

\DeclareMathOperator{\Hamming}{Hamming}
\DeclareMathOperator{\Span}{span}
\newcommand{\F}{\mathbb{F}}

\begin{document}
\maketitle

\noindent 
The basis vectors for $\Hamming(8,4)\subset \F_2^8$ are:\\
\begin{center}
$10000111$\\
$01001011$\\
$00101101$\\
$00011110$
\end{center}
Define $v_0 = 00011110$, $v_1=00101101$ and $V = \Span\{v_0,v_1\}\subset \F_2^8$.
Let's calculate $\theta\colon V\times V \to \F_2$
\marginnote{I do wonder how this relates to the quaternion group extension cocycle}
\`a la Theorem 10 of Griess\cite{Griess}. All equality is modulo 2.

\begin{itemize}

	\item[D0:] $\theta(v_0,v_0) = |v_0|/4,\quad \theta(0,0) = \theta(0,v_0) = \theta(v_0,0) = 0$.

	\item[D1:] For all $x \in \Span\{v_0\}$, 
	\begin{itemize}
		\item[] $\theta(v_1,x) = 0$, \marginnote{arbitrary choice, but with $\theta(v_1,0)=0$}
		\item[] $\theta(x,v_1) = |x\cap v_1|/2$.
	\end{itemize}

	\item[D2:] For all $x \in \Span\{v_0\}$,
	\begin{itemize}
	 	\item[] $\theta(v_1,v_1+x) = |v_1|/4$,
	 	\item[] $\theta(v_1 + x,v_1) = |v_1|/4 + |v_1\cap(v_1+x)|/2$.
	 \end{itemize} 
	 \item[D3:] For all $x,y\in \Span\{v_0\}$,
	 \begin{itemize}
	 	\item[] $\theta(v_1+x,v_1+y) = |y\cap(v_1+x)|/2 + |y\cap v_1|/2 + \theta(x,y) + |v_1|/4 + |v_1\cap(v_1+x)|/2$.
	 \end{itemize}
	 \item[D4:] For all $x,y\in \Span\{v_0\}$,
	 \begin{itemize}
	 	\item[] $\theta(v_1+x,y) = |v_1+x|/4 + \theta(v_1+x,v_1+x+y)$,
	 	\item[] $\theta(y,v_1+x) = |y\cap(v_1+x)|/2 + \theta(v_1+x,y)$.
	 \end{itemize}

\end{itemize}

\noindent 
The following gives $\theta(a,b)$, for $a,b\in V$.
\begin{center}
\begin{tabular}{c|cc|cc}
	\diagbox{a}{b} & $0$ & $v_0$ & $v_1$ & $v_0+v_1$ \\
	\hline
	$0$       		& $0$ & $0$ & $0$ & $0$ \\
	$v_0$     		& $0$ & $1$ & $1$ & $0$ \\
	\hline
	$v_1$     		& $0$ & $0$ & $1$ & $1$ \\
	$v_1+v_0$ 		& $0$ & $1$ & $0$ & $1$ \\
\end{tabular}
\end{center}

\noindent 
To define $\theta$ on $\Hamming(8,4)$, repeat, adding a new basis vector.

\bibliographystyle{alpha}
\bibliography{bib}
\end{document}