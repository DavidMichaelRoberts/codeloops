\documentclass[a4paper,debug]{tufte-handout}

\title{The II-cocycle: Electric Boogaloo}
\date{\today}
\author{DMR}

\usepackage{amsmath}

\hypersetup{
urlcolor = red,
colorlinks = true,
linkcolor = blue,
citecolor = blue,
linktocpage = true,
pdftitle = {The II-cocycle: Electric Boogaloo},
pdfauthor = {David M. Roberts},
bookmarksopen = false,
bookmarksopenlevel = 1,
unicode = true,
hypertexnames =false
}%

\DeclareMathOperator{\Hamming}{Hamming}
% \DeclareMathOperator{\dim}{dim}
\newcommand{\F}{\mathbb{F}}


\setlength{\parindent}{0pt}
\begin{document}
\maketitle

\noindent 
Recall the concept of \emph{code loop} from Griess\cite{Griess}
, and also that of a \emph{factor set}, 
which we will call a \emph{(twisted) cocycle}. 
Let $C \subset \F_2^{4n}$ be a doubly-even code.
The twisted cocycle identity for $\theta\colon C\times C \to \F_2$ is
\[
	\theta(x,y) + \theta(x,y+z) + \theta(x+y,z) + \theta(y,z) = |x\wedge y \wedge z|
\]
where $\wedge$ is co\"ordinate-wise multiplication, and $|\cdot|$ is the bit weight of a vector.

% \noindent
We can split $C$ as $V\oplus W$ and denote the restriction of $\theta$ to $(V\cup W)\times (V \cup W)$ by $\alpha$, which is a function
\[
	V\oplus V \cup W\oplus W \cup V\oplus W \cup W\oplus V \to \F_2.
\]
We can write $\theta$ entirely in terms of $\alpha$ as follows:
\begin{align*}
	\theta(v_1+w_1,v_2+w_2)	& = \alpha(v_1,v_2)  + \alpha(w_1,w_2) + \alpha(v_1,w_1) + \alpha(w_2,v_2) + \alpha(v_1+v_2,w_1+w_2)\\
							& + \frac12|v_2\wedge(w_1+w_2)| + |v_1\wedge v_2 \wedge (w_1+w_2)| + |w_1\wedge w_2 \wedge v_2| + |v_1\wedge w_1 \wedge (v_2 + w_2)|
\end{align*}

We can then present $\alpha$ graphically, with $(2^k + 2^l - 1)^2$ values, rather than presenting $\theta$, with $2^{2(k+l)}$ values, where $k=\dim V$ and $l=\dim W$. Together with the above equation this determines $\theta$ uniquely. 

For the Golay code, split into two subspaces of dimension 6, this means we only require the $127^2 = 16129$ values of $\alpha$ in order to determine the rest of the $2^{2\times 12} = 4096^2 = 16777216$ values of $\theta$.

%%%% This doesn't work!!
% There is nothing stopping us from iterating this procedure, and further splitting into three subspaces, say
% $C = U \oplus V \oplus W$, of dimensions $k,l$ and $m$ respectively, so that we only require $(2^k+2^l+2^m-2)^2$ values, rather than $2^{2(k+l+m)}$ values.
% In this instance we have a function
% \[
% 	\beta\colon (U\cup V \cup W) \times (U\cup V \cup W) \to \F_2
% \]
% and $\theta$ can be written as
% \begin{align*}
% 	\theta(u_1+v_1+w_1,u_2+v_2+w_2)	& = \alpha(u_1+v_1,u_2+v_2)  + \alpha(w_1,w_2) + \alpha(u_1+v_1,w_1) \\
% 									& + \alpha(w_2,u_2+v_2) + \alpha(u_1+u_2+v_1+v_2,w_1+w_2)\\
% 							& + \frac12|(u_2+v_2)\wedge(w_1+w_2)| + |(u_1+v_1)\wedge (u_2+v_2) \wedge (w_1+w_2)|\\
% 							& + |w_1\wedge w_2 \wedge (u_2+v_2)| + |(u_1+v_1)\wedge w_1 \wedge (u_2+v_2 + w_2)|
% \end{align*}
% where we have reused the formula involving $\alpha$ for the restriction $\theta\big|_{((U\oplus V) \cup W)^2}$ to the union of two subspaces. This can then be further broken down so as to use only $\beta$.

% \medskip
% {\footnotesize For the Golay code, we can split it into three subspaces each of dimension $4$, so that the $(3\times 2^4-2)^2 = 46^2 = 2116$ values of $\beta$ are enough to determine the twisted cocycle that determines the Parker loop, rather than all $2^{2\times 12} = 4096^2 = 16777216$ values. Note that $\alpha$ has $127^2 = 16129$ values in this case.}

% \newpage
% We will work out each term using $\alpha $in the previous expression in terms of $\beta$.
% \begin{align*}
% 	\alpha(u_1+v_1,u_2+v_2) & = \beta(u_1,u_2)  + \beta(v_1,v_2) + \beta(u_1,v_1) + \beta(v_2,u_2) + \beta(u_1+u_2,v_1+v_2)\\
% 							& + \frac12|u_2\wedge(v_1+v_2)| + |u_1\wedge u_2 \wedge (v_1+v_2)| + |v_1\wedge v_2 \wedge u_2| + |u_1\wedge v_1 \wedge (u_2 + v_2)|
% \end{align*}

\bibliographystyle{alpha}
\bibliography{bib}
\end{document}